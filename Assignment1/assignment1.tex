\documentclass[letterpaper,12pt]{article}
\usepackage{tabularx} % extra features for tabular environment
\usepackage{amsmath}  % improve math presentation
\usepackage{graphicx} % takes care of graphic including machinery
\usepackage[margin=1in,letterpaper]{geometry} % decreases margins
\usepackage{cite} % takes care of citations
\usepackage[final]{hyperref} % adds hyper links inside the generated pdf file
\hypersetup{
	colorlinks=true,       % false: boxed links; true: colored links
	linkcolor=blue,        % color of internal links
	citecolor=blue,        % color of links to bibliography
	filecolor=magenta,     % color of file links
	urlcolor=blue         
}
%\usepackage{blindtext}
%++++++++++++++++++++++++++++++++++++++++


\begin{document}

\title{
EE5609: MATRIX THEORY \\
    Assignment 1}
%\subtitle{Assignment 1}
\author{Sneha Konduru\\ee19acmtech11009}
\date{\today}
\maketitle




\section{Question (Lines and Planes. Problem 39)}

The line through the points \begin{pmatrix}
h\\
3
\end{pmatrix} and \begin{pmatrix}
4\\
1
\end{pmatrix}
intersects the line \begin{pmatrix}
7 & -9
\end{pmatrix} \textbf{X}=19 at right angle. Find the value of h.

\section{Solution}
In the problem statement it is given that, line passing through the points \begin{pmatrix}
h\\
1
\end{pmatrix}
and \begin{pmatrix}
4\\
q
\end{pmatrix}.
Slope of the line passing through the two points \begin{pmatrix}
x1\\
y1
\end{pmatrix} and \begin{pmatrix}
x2\\
y2
\end{pmatrix}
calculated by using the fallowing formula 

\begin{equation}
   \ slope= \frac{y1-y2}{x1-x2} 
\end{equation}

By using equation 1, slope calculated is 
\begin{equation}
\ m1= \frac{2}{h-4}
\end{equation}
%\newline
Slope of the straight line calculated by differentiating w.r.t x. Slope of the given straight line 7x-9y=19 is slope (m2)=$\dfrac{7}{9}$. 
\newline 
In the problem statement it is given that two straight lines are intersecting at right angles. When two straight lines are intersecting at right angles then their product of slopes is -1.

\begin{equation}
\ Product of slopes( m1 \times m2)=-1
\end{equation}
\newpage
\begin{equation}
   \frac{2}{h-4}  \times \dfrac{7}{9} =-1
\end{equation}
\begin{equation}
   \ h =\dfrac{22}{9} 
\end{equation}


\end{document}
