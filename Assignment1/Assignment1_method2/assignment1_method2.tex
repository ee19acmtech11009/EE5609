\documentclass[letterpaper,12pt]{article}
\usepackage{tabularx} % extra features for tabular environment
\usepackage{amsmath}  % improve math presentation
\usepackage{graphicx} % takes care of graphic including machinery
\usepackage[margin=1in,letterpaper]{geometry} % decreases margins
\usepackage{cite} % takes care of citations
\usepackage[final]{hyperref} % adds hyper links inside the generated pdf file
\hypersetup{
	colorlinks=true,       % false: boxed links; true: colored links
	linkcolor=blue,        % color of internal links
	citecolor=blue,        % color of links to bibliography
	filecolor=magenta,     % color of file links
	urlcolor=blue         
}
%\usepackage{blindtext}
%++++++++++++++++++++++++++++++++++++++++


\begin{document}

\title{
EE5609: MATRIX THEORY \\
    Assignment 1}
%\subtitle{Assignment 1}
\author{Sneha Konduru\\ee19acmtech11009}
\date{\today}
\maketitle




\section{Question (3.7.39)}

The line through the points \begin{pmatrix}
h\\
3
\end{pmatrix} and \begin{pmatrix}
4\\
1
\end{pmatrix}
intersects the line \begin{pmatrix}
7 & -9
\end{pmatrix} $\myvec{\textbf{x}}$=19 at right angle. Find the value of h.

\section{Solution}
Directional vector of line passing through points \textbf{A}=\begin{pmatrix}
h\\
3
\end{pmatrix} and \textbf{B}=\begin{pmatrix}
4\\
1
\end{pmatrix} is 
\begin{equation}
\textbf{P}=\textbf{B}-\textbf{A}  \tag{1}
\end{equation}
\begin{equation}
    \textbf{P}=\begin{pmatrix}
h-4\\
2
\end{pmatrix}  \tag{2}
\end{equation}
Directional vector of the line \begin{pmatrix}
a & b
\end{pmatrix}$\myvec{\textbf{x}}$=c is 
\begin{equation}
\textbf{Q}=\begin{pmatrix}
b\\
-a
\end{pmatrix}   \tag{3}
\end{equation}
From (3.7.39.3) direction vector of
line \begin{pmatrix}
7 & -9
 
\end{pmatrix}$\myvec{\textbf{x}}$=19 is 
\begin{equation}
\textbf{Q}=\begin{pmatrix}
-9\\
-7
\end{pmatrix}  \tag{4}
\end{equation}
If two straight lines intersects at right angles then inner product of their directional vectors \textbf{P} and \textbf{Q} is zero.
\begin{equation}
    {\textbf{P}^\textbf{T}}\textbf{Q}=0  \tag{5}
\end{equation}

\begin{equation}
    \begin{pmatrix}
h-4\\
2
\end{pmatrix}^\textbf{T} \begin{pmatrix}
-9\\ 
-7
\end{pmatrix}=0  \tag{6}
\end{equation}
\begin{equation}
  (h-4)(-9)+2(-7)=0   \tag{7}
\end{equation}
  
 \begin{equation}
  h=\dfrac{22}{9}  \tag{8}
\end{equation}
    
\end{document}
